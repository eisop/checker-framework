\htmlhr
\chapterAndLabel{Type inference}{type-inference}

This chapter is about tools that infer annotations for your
program's method signatures and fields, before you run a type-checker.
To learn about local type inference within a method body,
see Section~\ref{type-refinement}.

A typical workflow (Section~\ref{tips-about-writing-annotations}) is for a
programmer to first write annotations on method signatures and fields, then
run a type-checker.  Type inference performs the first step automatically
for you.  This saves time for programmers who would otherwise have to
understand the code, then write annotations manually.

Type inference outputs type qualifiers that are consistent with your
program's source code.  Your program still might not type-check if your
program contains a bug or contains tricky code that is beyond the
capabilities of the type checker.

The qualifiers are output into an annotation file.  They can be viewed and
adjusted by the programmer, can be used by tools such as the type-checker,
and can be inserted into the source code or the class file.

Inserting the inferred annotations into the program source code creates documentation in the form of type
qualifiers, which can aid programmer understanding and may make
type-checking warnings more comprehensible.
Storing annotations in side-files is more desirable if the program's source code cannot
be modified for some reason, if the typechecking is "one-off" (typechecking will
be done once and its results will be evaluated, but it will not be done
repeatedly), or if the set of annotations is extremely voluminous and would
clutter the code.


Type inference is most effective when you run it on a program rather than
on a library --- unless you also run it on an extensive test suite for the
library.  See Section~\ref{whole-program-inference-non-representative-uses}
for an explanation.


Type inference is costly:  it takes several times longer than
type-checking does.  However, it only needs to be run once, after which
you can use and possibly modify the results.


\sectionAndLabel{Type inference tools}{type-inference-tools}

This section lists tools that take a program and output a set of
annotations for it.
It first lists tools that work only for a single type system (but may do a
more accurate job for that type system)
then lists general tools that work for any type system.

\begin{description}
  \item[For the Nullness Checker:]
Section~\ref{nullness-inference} lists several tools that infer
annotations for the Nullness Checker.

\item[For the Purity Checker:]
If you run the Checker Framework with the \<-AsuggestPureMethods>
command-line option, it will suggest methods that can be marked as
\<@SideEffectFree>, \<@Deterministic>, or \<@Pure>; see
Section~\ref{type-refinement-purity}.

\item[WPI, for any type system:]
``Whole program inference'', or WPI, is distributed with the Checker
  Framework.  See Section~\ref{whole-program-inference}.

\item[CFI, for any type system:]
\href{https://github.com/opprop/checker-framework-inference}{``Checker
  Framework Inference''}, or CFI, is a type inference framework built on
a variant of the Checker Framework.  You need to slightly rewrite your type system to
work with CFI\@.  The
\ahreforurl{https://github.com/opprop/checker-framework-inference}{CFI
  repository} contains rewritten versions of some of
the type systems that are distributed with the Checker Framework.

\item[Cascade, for any type system:]
\href{https://github.com/reprogrammer/cascade/}{Cascade}~\cite{VakilianPEJ2015}
is an Eclipse plugin that implements interactive type qualifier inference.
Cascade is interactive rather than fully-automated:  it makes it easier for
a developer to insert annotations.
Cascade starts with an unannotated program and runs a type-checker.  For each
warning it suggests multiple fixes, the developer chooses a fix, and
Cascade applies it.  Cascade works with any checker built on the Checker
Framework.
You can find installation instructions and a video tutorial at \url{https://github.com/reprogrammer/cascade}.
% See last commit at https://github.com/reprogrammer/cascade/commits/master .
Cascade was last updated in November 2014, so it might or might not work for you.

\end{description}

Except for one of the nullness inference tools, all these
type inference tools are static analyses.  They analyze your program's
source code, but they do not run your program.


\sectionAndLabel{Whole-program inference}{whole-program-inference}

Whole-program inference
infers types for fields, method parameters, and method return types that do not
have a user-written qualifier (for the given type system).
The inferred type qualifiers are output into annotation files.
The inferred type is the most specific type that is compatible with all the
uses in the program.  For example, the inferred type for a field is the
least upper bound of the types of all the expressions that are assigned
into the field.

There are three scripts that you can use to run whole-program inference.
Each has advantages and disadvantages, discussed below:

\begin{itemize}
    \item
    To run whole-program inference on a single project without modifying its source code,
    use the \<wpi.sh> script (Section~\ref{wpi-one}). This script can automatically understand
    many Ant, Maven, and Gradle build files, so it requires little manual configuration.

    \item
    To run whole-program inference on many projects without modifying their source code
    (say, when running it on projects from GitHub), use the \<wpi-many.sh> script (Section~\ref{wpi-many}).
    This script can understand the same build files as \<wpi.sh>.

    \item
    If you want to insert the inferred annotations directly into a single
    project's source code, use the \<infer-and-annotate.sh> script (Section~\ref{wpi-insert}).
\end{itemize}

These type inference scripts appear in the \<checker/bin/> directory.
The remainder of this chapter describes them
(Sections~\ref{wpi-one}--\ref{wpi-insert}), then concludes with discussion
that applies to all of them.


\sectionAndLabel{Running whole-program inference on a single project}{wpi-one}

A typical invocation of \<wpi.sh> is

\begin{Verbatim}
  wpi.sh -- --checker nullness
\end{Verbatim}

The result is a set of log files placed in the \<dljc-out/> folder of the
target project. The results of type-checking with each candidate set of
annotations will be concatenated into the file \<dljc-out/wpi-stdout.log>; the final
results (i.e., those obtained using the most precise, consistent set of annotations)
will appear at the end of this file.
The inferred annotations appear in \<.ajava> files in a
temporary directory whose name appears in the \<dlcj-out/wpi-stdout.log> file;
you can find their location by examining the \<-Aajava> argument
to the last \<javac> command that was run. The annotation files generated in
each round of inference appear in directories labeled \<iteration0>, \<iteration1>, etc.

The \<wpi.sh> script is most useful when analyzing projects that follow the standard
conventions of their build system for single-module projects. When analyzing a project
that requires non-standard build system commands, use the \<-c> and \<-b> options to
override the defaults used by \<wpi.sh>. For example, suppose that you wanted to use
\<wpi.sh> to infer annotations for the Resource Leak Checker on
\href{https://github.com/apache/zookeeper}{Apache Zookeeper's}
\<zookeeper-server> module only. Without \<wpi.sh>, one might use the following command:
\<mvn -B --projects zookeeper-server --also-make install -DskipTests>. To use this command via
\<wpi.sh>, use the following:
\<sh wpi.sh -b "-B --projects zookeeper-server --also-make" -c "install -DskipTests" -- --checker resourceleak>.

The full syntax for invoking \<wpi.sh> is

\begin{Verbatim}
  wpi.sh [-d PROJECTDIR] [-t TIMEOUT] [-c COMPILATION_TARGET] [-b EXTRA_BUILD_ARGS] [-g GRADLECACHEDIR] -- [DLJC-ARGS]
\end{Verbatim}

Arguments in square brackets are optional.
Here is an explanation of the arguments:

\begin{description}
\item[-d PROJECTDIR]
  The top-level directory of the project.  It must contain an Ant, Gradle,
  or Maven buildfile. The default is the current working directory.

\item[-t TIMEOUT]
  The timeout for running the checker, in seconds.

\item[-c COMPILATION\_TARGET]
  The name(s) of the build system target(s) used to compile
  the target project. The default is chosen based on the build
  system (e.g., \<compileJava> for Gradle, \<compile> for Maven, etc.).
  This argument is passed directly to the build system when compiling the
  target project, so it can also include command-line arguments that only
  apply to the compilation targets (and not other targets like \<clean>);
  for general command-line arguments that apply to all targets, use the
  \<-b> option instead.
  When running WPI on a single module of a multi-module project,
  you might want to use this option (possibly in combination with \<-b>, below),
  depending on the setup of the target project.
  The argument may contain spaces and is re-tokenized by the shell.

\item[-b EXTRA\_BUILD\_ARGS]
  Extra arguments to pass to the build script invocation. This argument
  will be passed to compilation tasks, such as
  \<ant compile>, \<gradle compileJava>, or \<mvn compile>.
  The main difference between this option and \<-c> is that the values
  of the \<-b> option are also passed to other, non-compilation build system commands, such as
  \<ant clean>, \<gradle clean>, or \<mvn clean>.
  % An alternative would be to permit it to be passed multiple times, and
  % not re-tokenize it.
  The argument may contain spaces and is re-tokenized by the shell.

\item[-g GRADLECACHEDIR]
  The directory to use for the \<-g> option to Gradle (the Gradle home
  directory). This option is ignored if the target project does not
  build with Gradle. The default is \<.gradle> relative to the target
  project (i.e., each target project has its own Gradle home). This default
  is motivated by
  \href{https://github.com/gradle/gradle/issues/1319}{Gradle issue \#1319}.

\label{DLJC-ARGS}
\item[DLJC-ARGS]
  Arguments that are passed directly to
  \href{https://github.com/kelloggm/do-like-javac}{do-like-javac}'s
  \<dljc> program without
  modification.  One argument is required: \<-\relax-checker>, which indicates
  what type-checker(s) to run (in the format described in Section~\ref{shorthand-for-checkers}).

  The \ahreforurl{https://github.com/kelloggm/do-like-javac}{documentation of do-like-javac}
  describes the other commands that its WPI tool supports. Notably, to pass checker-specific
  arguments to invocations of \<javac>,
  use the \<-\relax-extraJavacArgs> argument to \<dljc>. For example, to use the \<-AignoreRangeOverflow>
  option for the Constant Value Checker (\chapterpageref{constant-value-checker}) when running
  inference, you would add \<-\relax-extraJavacArgs='-AignoreRangeOverflow'> anywhere after the \<-\relax->
  argument to \<wpi.sh>.
\end{description}

You may need to wait a few minutes for the command to complete.


\subsectionAndLabel{Requirements for whole-program inference scripts}{wpi-shared-requirements}

The requirements to run \<wpi.sh> and \<wpi-many.sh> are the same:

\begin{itemize}
\item The project on which inference is run must contain an Ant, Gradle,
  or Maven buildfile that compiles the project.
\item At least one of the \verb|JAVA_HOME|,
  \verb|JAVA8_HOME|, \verb|JAVA11_HOME|, \verb|JAVA17_HOME|, or \verb|JAVA21_HOME|
  environment variables must be set.
\item If set, the \verb|JAVA_HOME| environment variable must point to a
  Java 8, 11, 17, 20, or 21 JDK.
\item If set, the \verb|JAVA8_HOME| environment variable must point to a Java 8 JDK.
\item If set, the \verb|JAVA11_HOME| environment variable must point to a Java 11 JDK.
\item If set, the \verb|JAVA17_HOME| environment variable must point to a Java 17 JDK.
\item If set, the \verb|JAVA20_HOME| environment variable must point to a Java 20 JDK.
\item If set, the \verb|JAVA21_HOME| environment variable must point to a Java 21 JDK.
\item \<CHECKERFRAMEWORK> environment variable must point to a built copy of the Checker Framework.
\item If set, the \verb|DLJC| environment variable must point to a copy of the \<dljc> script
from \ahreforurl{https://github.com/kelloggm/do-like-javac}{do-like-javac}. (If this variable is not
set, the WPI scripts will download this dependency automatically.)
\item Other dependencies:
  ant,
  awk,
  curl,
  git,
  gradle,
  mvn,
  python3 (for dljc),
  wget.

  Python2.7 modules:
  subprocess32.
\end{itemize}



\sectionAndLabel{Running whole-program inference on many projects}{wpi-many}

The requirements to run \<wpi.sh> and \<wpi-many.sh> are the same. See Section~\ref{wpi-shared-requirements}
for the list of requirements.

To run an experiment on many projects:
\begin{enumerate}
\item Use \<query-github.sh> to search GitHub for candidate repositories.
File \<docs/examples/wpi-many/securerandom.query> is an example query, and file
\<docs/manual/securerandom.list> is the standard output
created by running \<query-github.sh securerandom.query 100>. If you do
not want to use GitHub, construct a file yourself that matches the format of
the file \<securerandom.list>.

\item Use \<wpi-many.sh> to run whole-program inference on multiple
Ant, Gradle, or Maven projects.  You provide it a file in which each line
is ``GitHub repository URL'' ``git hash'' (with no quotes, and with
whitespace between them).
\begin{itemize}
\item If you are using a checker that is distributed with the Checker
Framework, use \<wpi-many.sh> directly.
\item If you are using a checker that is not distributed with the Checker
Framework (also known as a "custom checker"), file
\<docs/examples/wpi-many/wpi-many-custom-checker-example.sh> is a no-arguments
script that serves as an example of how to use \<wpi-many.sh>.
\end{itemize}

Log files are copied into a results directory.
For a failed run, the log file indicates the reason that WPI could not
be run to completion on the project.
For a successful run, the log file indicates whether the project was verified
(i.e., no errors were reported), or whether the checker issued warnings
(which might be true positive or false positive warnings).

\item Use \<wpi-summary.sh> to summarize the logs in the output results directory.
Use its output to guide your analysis of the results of running \<wpi-many.sh>:
you should manually examine the log files for the projects that appear in the
"results available" list it produces. This list is the list of every project
that the script was able to successfully run WPI on.  (This does not mean
that the project type-checks without errors afterward, or even typechecks at
all --- just that the Checker Framework attempted to typecheck the project and
some output was produced.)

\item (Optional) Add annotations that WPI does not infer, to eliminate
  false positive warnings. There are two ways you can add annotations to a target
  program:
  \begin{itemize}
  \item
    Fork the project and add the annotations directly to the project's
    source code, and add a dependency on
    \<org.checkerframework:checker-qual> to the project's build
    system. This approach is the most difficult, but has the advantage that
    the checker will attempt to verify any annotations you add.
  \item Add the annotations in a stub file, creating the \<.astub> file as
    a copy of the \<.java> file.  When you run the checker, supply the
    \<-AmergeStubsWithSource> and \<-Astubs=...> command-line arguments.
  \end{itemize}

\end{enumerate}

A typical invocation is

\begin{Verbatim}
wpi-many.sh -o outdir -i /path/to/repo.list -t 7200 -- --checker optional
\end{Verbatim}

The \<wpi-many.sh> script takes the following command-line arguments.
The \<-o> and \<-i> arguments are mandatory.
An invocation should also include \<-- [\emph{DLJC-ARGS}]> at the end;
\emph{DLJC-ARGS} is documented in Section~\ref{DLJC-ARGS}.


\begin{description}
\item[-o outdir]
  run the experiment in the \<\emph{outdir}> directory, and place the results in
  the \<\emph{outdir}-results> directory. Both will be created if they do not
  exist.  The directory may be specified as an absolute or relative path.

\item[-i infile]
  Read the list of repositories to use from the file infile.
  % The need to be an absolute pathname is a bug in wpi-many.sh that should be fixed.
  The file must be specified as an absolute, not relative, path.
  Each line
  should have 2 elements, separated by whitespace:
  \begin{enumerate}
  \item
    The URL of the git repository on GitHub. The URL must be of the form
    https://github.com/username/repository .  The script is reliant on the
    number of slashes, so excluding ``https://'' is an error.
  \item The commit hash to use.
  \end{enumerate}

\item[-t timeout]
  The timeout for running the checker on each project, in seconds.

\item[-g GRADLECACHEDIR]
  The directory to use for the \<-g> option to Gradle (the Gradle home
  directory). This option is ignored if the target project does not
  build with Gradle. The default is \<.gradle> relative to the target
  project (i.e., each target project has its own Gradle home). This default
  is motivated by
  \href{https://github.com/gradle/gradle/issues/1319}{Gradle issue \#1319}.

\item[-s]
  If this flag is present, then projects which are not buildable --- for which
  no supported build file is present or for which running the standard build
  commands fail --- are skipped on future runs but are \emph{not} deleted immediately
  (such projects are deleted immediately if this flag is not present). This flag is useful
  if you intend to run \<wpi-many.sh> several times on the same set of repositories
  (for example, during checker development), to avoid re-downloading unusable projects.

\end{description}


\sectionAndLabel{Whole-program inference that inserts annotations into source code}{wpi-insert}

\begin{sloppypar}
To use this version of whole-program inference, make sure that
\<insert-annotations-to-source>, from the Annotation File Utilities project,
is on your path (for example, its directory is in the \<\$PATH> environment variable).
Then, run the script \<checker-framework/checker/bin/infer-and-annotate.sh>.
Its command-line arguments are:
\end{sloppypar}

\begin{enumerate}
\item Optional: Command-line arguments to
  \href{https://eisop.github.io/afu/#insert-annotations-to-source}{\<insert-annotations-to-source>}.
\item Processor's name.
\item Target program's classpath.  This argument is required; pass "" if it
  is empty.
\item Optional: Extra processor arguments which will be passed to the checker, if any.
  You may supply any number of such arguments, or none.  Each such argument
  must start with a hyphen.
\item Optional: Paths to \<.jaif> files used as input in the inference
    process.
\item Paths to \<.java> files in the program.
\end{enumerate}

% TODO: Change the example project that is being annotated, since plume-lib is now deprecated.
For example, to add annotations to the \<plume-lib> project:
\begin{Verbatim}
git clone https://github.com/mernst/plume-lib.git
cd plume-lib
make jar
$CHECKERFRAMEWORK/checker/bin/infer-and-annotate.sh \
    "LockChecker,NullnessChecker" java/plume.jar:java/lib/junit-4.12.jar:$JAVA_HOME/lib/tools.jar \
    $(find java/src/plume/ -name "*.java")
# View the results
git diff
\end{Verbatim}

You may need to wait a few minutes for the command to complete.
You can ignore warnings that the command outputs while it tries different
annotations in your code.

It is recommended that you run \<infer-and-annotate.sh> on a copy of your
code, so that you can see what changes it made and so that it does not
change your only copy.  One way to do this is to work in a clone of your
repository that has no uncommitted changes.


\sectionAndLabel{Inference results depend on uses in your program or test suite}{whole-program-inference-non-representative-uses}

Type inference outputs the most specific type qualifiers that are
consistent with all the source code it is given.
(Section~\ref{whole-program-inference-ignores-some-code}
explains when type inference ignores some code.)
This may be different than the specification the programmer had in mind
when writing tho code.
If the program uses a method or field in a limited way, then the inferred
annotations will be legal for the program as
currently written but may not be as general as possible and may not
accommodate future program changes.

Here are some examples:

\begin{itemize}
\item
Suppose that your program (or test suite) currently calls
method \<m1> only with non-null
arguments.  The tool will infer that \<m1>'s parameter has
\<@NonNull> type.  If you had intended the method to be able to
take \<null> as an argument and you later add such a call, the type-checker
will issue a warning because the inferred \<@NonNull>
annotation is inconsistent with the new call.

\item
If your program (or test suite) passes only \<null> as an argument, the
inferred type will be the bottom type, such as \<@GuardedByBottom>.

\item
Suppose that method \<m2> has no body, because it is defined in an interface or
abstract class.
Type inference can still infer types for its signature, based on the
overriding implementations.
If all the methods that override \<m2> return a non-null value, type
inference will infer that \<m2>'s return type has \<@NonNull> type, even if
some other overriding method is allowed to return
\<null>.

\end{itemize}

If the program contains erroneous calls, the
inferred annotations may reflect those errors.
Suppose you intend method \<m3> to be called with
non-null arguments, but your program contains an error and one of the calls
to \<m3> passes \<null> as the argument.  Then the tool will infer that
\<m3>'s parameter has \<@Nullable> type.

If you run whole-program inference on a library that contains mutually
recursive routines, and there are no non-recursive calls to the routines,
then whole-program inference may run a long time and eventually produce
incorrect results.  In this case, write type annotations on the formal
parameters of one of the routines.

Whole-program inference is a ``forward analysis''.
% This might change in the future.
It determines a method parameter's type
annotation based on what arguments are passed to the method but not on how the
parameter is used within the method body.
It determines a method's return type based on code in the method body but
not on uses of the method return value in client code.


\subsectionAndLabel{Whole-program inference ignores some code}{whole-program-inference-ignores-some-code}

Whole-program inference ignores code within the scope of a
\<@SuppressWarnings> annotation with an appropriate key
(Section~\ref{suppresswarnings-annotation}).  In particular, uses within
the scope do not contribute to the inferred type, and declarations within
the scope are not changed.  You should remove \<@SuppressWarnings> annotations
from the class declaration of any class you wish to infer types for.

As noted above, whole-program inference generalizes from invocations of methods and
assignments to fields.  If a field is set via
reflection (such as via injection), there are no explicit assignments to it
for type inference to generalize from, and type inference will produce
an inaccurate result.  There are two ways to make whole-program inference
ignore such a field.
%
(1)
You probably have an annotation such as
\javaeejavadocanno{javax/inject/Inject.html}{Inject}
or
\href{https://types.cs.washington.edu/plume-lib/api/plume/Option.html}{\<@Option>}
that indicates such fields.  Meta-annotate the declaration of the \<Inject>
or \<Option> annotation with
\refqualclass{framework/qual}{IgnoreInWholeProgramInference}.
%
(2)
Annotate the field to be ignored with
\refqualclass{framework/qual}{IgnoreInWholeProgramInference}.

Whole-program inference, for a type-checker other than the Nullness Checker,
ignores assignments and pseudo-assignments where the right-hand-side is the \<null> literal.


\subsectionAndLabel{Manually checking whole-program inference results}{whole-program-inference-manual-checking}

With any type inference tool, it is a good idea to manually examine the
results.  This can help you find bugs in your code or places where type
inference inferred an overly-precise result.
You can correct the inferred results manually, or you can
add tests that pass additional values and then re-run inference.

When arguments or assignments are literals, whole-program inference
commonly infers overly precise type annotations, such as \<@Interned> and
\<@Regex> annotations when the analyzed code only uses a constant string.

When an annotation is inferred for a \emph{use} of a type variable,
you may wish to move the annotation
to the corresponding upper bounds of the type variable \emph{declaration}.


\sectionAndLabel{How whole-program inference works}{how-whole-program-inference-works}

This section explains how the \<wpi.sh> and \<infer-and-annotate.sh> scripts work.  If you
merely want to run the scripts and you are not encountering trouble, you can
skip this section.

Each script repeatedly runs the checker with an \<-Ainfer=> command-line option to infer
types for fields and method signatures.  The output of this step
is a \<.jaif> (for \<infer-and-annotate.sh>) or \<.ajava> (for \<wpi.sh>) file that records the inferred types.
Each script adds the inferred annotation to the next run, so that the checker takes them into
account (and checks them). \<wpi.sh> does this by updating the set of \<.ajava> files that are passed
to the checker via the \<-Aajava> command-line argument;
\<infer-and-annotate.sh> inserts the inferred annotations in the program using the
\ahreforurl{https://eisop.github.io/afu/}{Annotation File Utilities}.

On each
iteration through the process, there may be new annotations in the \<.jaif> or \<.ajava>
files, and some type-checking errors may be eliminated (though others might
be introduced).
The process halts when there are no more changes to the inference results,
that is, the \<.jaif> or \<.ajava> files are unchanged between two runs.

When the type-checker is run on the program with the final annotations
inserted, there might still be errors.  This may be because the tool did
not infer enough annotations, or because your program cannot typecheck
(either because contains a defect, or because it contains subtle code that
is beyond the capabilities of the type system).
However, each of the inferred annotations is sound, and this reduces your
manual effort in annotating the program.

The iterative process is required because type-checking is modular:  it
processes each class and each method only once, independently.  Modularity
enables you to run type-checking on only part of your program, and it makes
type-checking fast.  However, it has some disadvantages:
\begin{itemize}
\item
  The first run of the type-checker cannot take advantage of whole-program
  inference results because whole-program inference is only complete at the
  end of type-checking, and modular type-checking does not revisit any
  already-processed classes.
\item
  Revisiting an
  already-processed class may result in a better estimate.
\end{itemize}


\sectionAndLabel{Type inference compared to other whole-program analyses}{type-inference-vs-whole-program-analysis}

There exist monolithic whole-program analyses that run without requiring any
annotations in the source code.  An advantage of such a tool is that the
programmer never needs to write any type annotations.

Running a whole-program inference tool, then running a type-checker, has
some benefits:
\begin{itemize}
\item
  The type qualifiers act as machine-checked documentation,
  which can aid programmer understanding.
\item
  Error messages may be more comprehensible.  With a monolithic
  whole-program analysis, error messages can be obscure, because the
  analysis has already inferred (possibly incorrect) types for a number of
  variables.
\item
  Errors are localized.  A change to one part of the program does not lead
  to an error message in a far-removed part of the program.
\item
  Type-checking is modular, which can be faster than re-doing a
  whole-program analysis every time the program changes.
\end{itemize}



%%  LocalWords:  Ainfer java jaif plugin classpath m2 m1 multi javax CFI
%%  LocalWords:  AsuggestPureMethods CHECKERFRAMEWORK GuardedByBottom dljc
%%  LocalWords:  IgnoreInWholeProgramInference typechecking Inference'' m3
% LocalWords:  PROJECTDIR awk gradle mvn python3 wget subprocess32 github
% LocalWords:  securerandom astub typecheck
