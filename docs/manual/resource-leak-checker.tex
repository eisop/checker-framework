\htmlhr
\chapterAndLabel{Resource Leak Checker for must-call obligations}{resource-leak-checker}

The Resource Leak Checker guarantees that the program fulfills every object's
must-call obligations before the object is de-allocated.

A resource leak occurs when a program does not explicitly dispose of some finite
underlying resource, such as a socket, file descriptor, or database connection.  To dispose
of the resource, the program should call some method on an object.
(De-allocating or garbage-collecting the object is not sufficient.)  For
example, the program must call \<close()> on every object that implements the
interface \<java.io.Closeable>.

The Resource Leak Checker can check any property of the form ``the programmer
must call each method in a set of methods \emph{M} at least once
on object \emph{O} before \emph{O} is de-allocated''.  For resource leaks,
by default \emph{M} is the set containing
\<close()> and \emph{O} is any object that implements the interface
\<java.io.Closeable>.  You can extend this guarantee to other types and methods
by writing \<@MustCall> or \<@InheritableMustCall> annotations, as described in
Section~\ref{must-call-annotations}.

The Resource Leak Checker works in three stages:
\begin{enumerate}
\item The Must Call Checker (\chapterpageref{must-call-checker})
  over-approximates each expression's must-call methods as a
  \refqualclass{checker/mustcall/qual}{MustCall} type.
\item The Called Methods Checker (\chapterpageref{called-methods-checker})
  under-approximates each expression's definitely-called methods as a
  \refqualclass{checker/calledmethods/qual}{CalledMethods} type.
\item When any program element goes out of scope (i.e., it is ready to be
  de-allocated), the Resource Leak Checker compares the types
  \<@MustCall(\emph{MC})> and \<@CalledMethods(\emph{CM})>.  It reports an error
  if there exists some method in \emph{MC} that is not in \emph{CM}.
\end{enumerate}

The paper
``Lightweight and Modular Resource Leak Verification''~\cite{KelloggSSE2021} (ESEC/FSE 2021,
\myurl{https://homes.cs.washington.edu/~mernst/pubs/resource-leak-esecfse2021-abstract.html})
gives more details about the Resource Leak Checker.


\sectionAndLabel{How to run the Resource Leak Checker}{resource-leak-run-checker}

Run one of these lines:

\begin{Verbatim}
javac -processor resourceleak MyFile.java ...
javac -processor org.checkerframework.checker.resourceleak.ResourceLeakChecker MyFile.java ...
\end{Verbatim}

The Resource Leak Checker supports all the command-line arguments
listed in Section~\ref{called-methods-run-checker} for
the Called Methods Checker, plus two others:

\begin{description}
\item[\<-ApermitStaticOwning>]
  See Section~\ref{resource-leak-owning-fields}.
\item[\<-ApermitInitializationLeak>]
  See Section~\ref{resource-leak-field-initialization}.
\end{description}

If you are running the Resource Leak Checker, then there is no need to run
the Must Call Checker (\chapterpageref{must-call-checker}), because the
Resource Leak Checker does so automatically.


\sectionAndLabel{Resource Leak Checker annotations}{resource-leak-annotations}

The Resource Leak Checker relies on the type qualifiers of two other checkers:
the Must Call Checker (Section~\ref{must-call-annotations}) and
the Called Methods Checker (Section~\ref{called-methods-spec}). You might need
to write qualifiers from either type hierarchy. The most common annotations from
these checkers that you might need to write are:

\begin{description}

\item[\refqualclasswithparams{checker/mustcall/qual}{MustCall}{String[] value}]
for example on an element with compile-time type \<Object> that might contain a \<Socket>.
See Section~\ref{must-call-annotations}.

\item[\refqualclasswithparams{checker/mustcall/qual}{InheritableMustCall}{String[] value}]
on any classes defined in your program that have must-call obligations. See Section~\ref{must-call-on-class}.

\item[\refqualclass{checker/calledmethods/qual}{EnsuresCalledMethods}] on a method in
your code that fulfills a must-call obligation of one of its parameters or of a field.
See Section~\ref{called-methods-ensurescalledmethods}.

\end{description}

The Resource Leak Checker supports annotations that express
aliasing patterns related to resource leaks:

\begin{description}

\item[\refqualclass{checker/mustcall/qual}{Owning}]
\item[\refqualclass{checker/mustcall/qual}{NotOwning}]
  expresses ownership.  When two aliases exist to the same Java object,
  \<@Owning> and \<@NotOwning> indicate which of the two is responsible for
  fulfilling must-call obligations.  For details, see
  Section~\ref{resource-leak-ownership}.

\item[\refqualclass{checker/mustcall/qual}{MustCallAlias}]
  represents a ``resource-aliasing'' relationship.  Resource aliases are
  distinct Java objects that control the same resource(s):
  fulfilling the must-call obligations of one object also
  fulfills the obligations of the other object.  For details,
  see Section~\ref{resource-leak-resource-alias}.

\end{description}

The Resource Leak Checker also supports an annotation to permit re-assigning
fields or re-opening resources:

\begin{description}

\item[\refqualclasswithparams{checker/mustcall/qual}{CreatesMustCallFor}{String value}]
  is a declaration annotation that indicates that after a call to a method
  with this annotation none of the must-call obligations of the in-scope, owning expression
  listed in \<value> have been met.
  In other words, the annotated method ``resets'' the must-call obligations of the expression.
  Multiple \<@CreatesMustCallFor>
  annotations can be written on the same method.  For more details
  on how to use this annotation to permit re-assignment of owning
  fields or the re-opening of resources, see Section~\ref{resource-leak-createsmustcallfor}.

\end{description}


\sectionAndLabel{Example of how safe resource usage is verified}{resource-leak-example}

Consider the following example of safe use of a \<Socket>, in which the comments indicate the
inferred Must Call and Called Methods type qualifiers for \<s>:
\begin{verbatim}
{
  Socket s = null;
  // 1. @MustCall({}) @CalledMethodsBottom
  try {
    s = new Socket(myHost, myPort);
    // 2. @MustCall("close") @CalledMethods({})
  } catch (Exception e) {
    // do nothing
  } finally {
    if (s != null) {
      s.close();
      // 3. @MustCall("close") @CalledMethods("close")
    } else {
      // do nothing
      // 4. @MustCall("close") @CalledMethodsBottom
    }
    // 5. @MustCall("close") @CalledMethods("close")
  }
  // 6. @MustCall("close") @CalledMethods("close")
}
\end{verbatim}

At point 1, \<s>'s type qualifiers are the type qualifiers of \<null>:
\<null> has no must-call obligations (\<@MustCall(\{\})>),
and methods cannot be called on it (\<@CalledMethodsBottom>).

At point 2, \<s> is a new \<Socket> object, which
has a must-call obligation (\<@MustCall("close")>)
and has had no methods called on it (\<@CalledMethods(\{\})>).

At point 3, \<close()> has definitely been called on \<s>, so
\<s>'s Called Methods type is updated. Note that the Must Call type
does not change.

At point 4, \<s> is definitely \<null> and its type is adjusted accordingly.

At point 5, \<s>'s type is the least upper bound of the types at points 3
and 4.

At point 6, \<s> goes out of scope.  The Resource Leak Checker reports a
\<required.method.not.called> error if the Must Call set contains any
element that the Called Methods set does not.


\sectionAndLabel{Aliased references and ownership transfer}{resource-leak-ownership}

Resource leak checking is complicated by aliasing.  Multiple expressions
may evaluate to the same Java object, but each object only needs to be
closed once.  (Section~\ref{resource-leak-resource-alias} describes a
related situation called ``resource aliasing'', when multiple Java objects
refer to the same underlying resource.)

For example, consider the following code that safely closes a \<Socket>:

\begin{verbatim}
  void example(String myHost, int myPort) {
    Socket s = new Socket(myHost, myPort);
    closeSocket(s);
  }
  void closeSocket(@Owning @MustCall("close") Socket t) {
    t.close();
  }
\end{verbatim}

There are two aliases for a socket object: \<s> in \<example()> and \<t> in
\<closeSocket()>.  Ordinarily, the Resource Leak Checker requires that
\<close()> is called on every expression of type \<Socket>, but that is not
necessary here.  The Resource Leak Checker should not warn when
\<s> goes out of scope in \<example()>, because \<closeSocket()> takes ownership
of the socket --- that is, \<closeSocket()> takes responsibility for closing
it. The \<@Owning> annotation on \<t>'s declaration expresses this fact; it
tells the Resource Leak Checker that \<t> is the reference that must be
closed, and its alias \<s> need not be closed.

Constructor returns are always \<@Owning>.
Method returns default to \<@Owning>,
and parameters and fields default to \<@NotOwning>. This treatment of parameter and
return types ensures sound handling of unannotated third-party libraries: any
object returned from such a library will be tracked by default, and the checker
never assumes that passing an object to an unannotated library will satisfy its obligations.

\<@Owning> and \<@NotOwning> always \emph{transfer} must-call obligations: must-call
obligations are conserved (i.e., neither created nor destroyed) by ownership annotations.
Writing \<@Owning> or \<@NotOwning> can never make the checker
unsound:  a real warning can never be hidden by them.
As with any annotation, incorrect or missing annotations can lead to false positive warnings.


\subsectionAndLabel{Owning fields}{resource-leak-owning-fields}

Unannotated fields are treated as non-owning.

For \textbf{final, non-static owning fields},
the Resource Leak Checker enforces the ``resource acquisition is
initialization (RAII)'' programming idiom.  Some
destructor-like method \<d()> must satisfy the field's must-call obligation
(and this fact must be expressed via a \<@EnsuresCalledMethods> annotation on \<d()>),
and the enclosing class must have a \<@MustCall("d")> obligation to
ensure the destructor is called. In addition to the \<@EnsuresCalledMethods> annotation,
which guarantees that the field(s) it references have their must-call obligations satisfied
on non-exceptional paths, the Resource Leak Checker requires those fields to have their must-call
obligations satisfied on all paths in (only) the destructor, and will issue a \<destructor.exceptional.postcondition>
error if they are not satisfied. Resolve this error by ensuring that the required methods are called
on all exceptional paths.

\textbf{Non-final, non-static owning fields} usually require one or more \<@CreatesMustCallFor> annotations
when they might be re-assigned. See Section~\ref{resource-leak-createsmustcallfor} for
more details on how to annotate a non-final, non-static owning field.

An assignment to a \textbf{static owning field} does not satisfy a
must-call obligation; for example,

\begin{smaller}
\begin{Verbatim}
  static @Owning PrintWriter debugLog = new PrintWriter("debug.log");
\end{Verbatim}
\end{smaller}

\noindent
The Resource Leak Checker issues a warning about every assignment of an
object with with a must-call annotation into a static owning field,
indicating that the obligation of the field's content might not be
satisfied.  When those fields are used throughout execution, until the
program exits, there is no good place to dispose of them, so these warnings
might not be useful.  The \<-ApermitStaticOwning> command-line argument
suppresses warnings related to static owning fields.  This can help in
checking legacy code.  It permits only a small number of resource retained
throughout execution, related to the number of such fields and assignments
to them.


\sectionAndLabel{Resource aliasing}{resource-leak-resource-alias}

A \emph{resource alias} set is a set of Java objects that
correspond to the same underlying system resource.
Calling a must-call method on any member of a resource-alias set
fulfills that obligation for all members of the set.
Members of the set may have different Java types.

Programmers most often encounter resource aliasing when using \emph{wrapper types}.
For example, the Java \<Buffered\-Output\-Stream> wrapper adds buffering to a
delegate stream.
The wrapper's \<close()> method invokes \<close()> on the delegate.  Calling
\<close()> on either object has the same effect:  it closes the underlying resource.

A resource aliasing relationship is expressed in source code via a pair of \<@MustCallAlias> annotations:
one on a parameter of a method or constructor, and another on its return type.
For example, the annotated JDK contains this constructor of \<BufferedOutputStream>:
\begin{Verbatim}
@MustCallAlias BufferedOutputStream(@MustCallAlias OutputStream out);
\end{Verbatim}

When a pair of \<@MustCallAlias> annotations is written on a method or constructor \<m>'s return type
and its parameter \<p>, the Resource Leak Checker requires one of the following:
\begin{enumerate}
\item \<p> is passed to another method or constructor (including \<super>) in a
  \<@MustCallAlias> position, and \<m> returns that method's result, or
\item \<p> is stored in an \<@Owning> field of the enclosing class.
\end{enumerate}


\sectionAndLabel{Creating obligations (how to re-assign a non-final owning field)}{resource-leak-createsmustcallfor}

Consider a class that has must-call obligations; that is, the class
declaration is annotated with \<@MustCall(...)>.
Every constructor implicitly creates obligations for the newly-created object.
Non-constructor methods may also create obligations
when re-assigning non-final owning fields or allocating
new system-level resources.

A post-condition annotation,
\<@CreatesMustCallFor>,
indicates for which expression an obligation is created.
If you write \<@CreatesMustCallFor(>\emph{T}\<)> on a method \emph{N} that
overrides a method \emph{M}, then \emph{M} must also be annotated as
\<@CreatesMustCallFor(>\emph{T}\<)>.  (\emph{M} may also have other
\<@CreatesMustCallFor> annotations that \emph{N} does not.)

\<@CreatesMustCallFor> allows the Resource Leak Checker to verify uses of non-final fields
that contain a resource, even if they are re-assigned. Consider
the following example:

\begin{verbatim}
  @MustCall("close") // default qualifier for uses of SocketContainer
  class SocketContainer {
    private @Owning Socket sock;

    public SocketContainer() { sock = ...; }

    void close() { sock.close() };

    @CreatesMustCallFor("this")
    void reconnect() {
      if (!sock.isClosed()) {
        sock.close();
      }
      sock = ...;
    }
  }
\end{verbatim}

In the lifetime of a \<SocketContainer> object, \<sock>
might be re-assigned arbitrarily many times: once at each
call to \<reconnect()>. This code is safe, however: \<reconnect()>
ensures that \<sock> is closed before re-assigning it.

Sections~\ref{resource-leak-createsmustcallfor-callsite}
and~\ref{resource-leak-createsmustcallfor-declaration}
explain how the Resource Leak Checker verifies uses and declarations of
methods annotated with \<@CreatesMustCallFor>.


\subsectionAndLabel{Requirements at a call site of a \<@CreatesMustCallFor> method}{resource-leak-createsmustcallfor-callsite}

At a call site to a method annotated as
\<@CreatesMustCallFor(>\emph{expr}\<)>, the Resource Leak Checker:
\begin{enumerate}
\item
  Treats any existing \<@MustCall> obligations of \emph{expr} as \emph{satisfied},
\item
  Creates a fresh obligation to check, as if \emph{expr} was assigned to a newly-allocated
  object (i.e. as if \emph{expr} were a constructor result).
\item
  Un-refines the type in the Called Methods Checker's type hierarchy for \emph{expr} to
  \<@CalledMethods(\{\})>, if it had any other Called Methods type.
\item
  Requires that the expression corresponding to \emph{expr} (that is, \emph{expr}
  viewpoint-adapted to the method call site) is owned; that is, it is
  annotated or defaulted as \<@Owning>.  Otherwise, the checker
  will issue a \<reset.not.owning> error at the call-site. You can avoid this
  error by extracting \emph{expr} into a new local variable (because
  locals are \<@Owning> by default) and replacing all instances of \emph{expr}
  in the call with references to the new local variable.
\end{enumerate}

Treating the obligation before the call as satisfied is sound: the
checker creates a new obligation for calls to \<@CreatesMustCallFor> methods,
and the Must Call Checker (\chapterpageref{must-call-checker}) ensures the
\<@MustCall> type for the target expression will have a \emph{superset} of any methods
present before the call. Intuitively, calling an \<@CreatesMustCallFor> method
``resets'' the obligations of the target expression, so whether they were satisfied before
the call or not is irrelevant.

If an \<@CreatesMustCallFor>
method \emph{n} is invoked within a method \emph{m} that has an \<@CreatesMustCallFor> annotation,
and the \<@CreatesMustCallFor> annotations on \emph{n} and \emph{m} have
the same target expression---imposing the obligation produced by calling \emph{n} on the caller of \emph{m}---then
the newly-created obligation is treated as satisfied immediately
at the call-site of \emph{n} in the body of \emph{m} (because it is imposed at call-sites of \emph{m}
instead).


\subsectionAndLabel{Requirements at a declaration of a \<@CreatesMustCallFor> method}{resource-leak-createsmustcallfor-declaration}

Any method that re-assigns a non-final, owning field of some object \<obj>
must be annotated \<@CreatesMustCallFor("obj")>.
Other methods may also be annotated with \<@CreatesMustCallFor>.

The Resource Leak Checker enforces two rules to ensure that re-assignments
to non-final, owning fields (like \<sock> in method \<reconnect> above) are
sound:
\begin{itemize}
\item any method that re-assigns a non-final, owning field of an object
  must be annotated with a \<@CreatesMustCallFor> annotation
  whose expression is a reference to that object.
\item when a non-final, owning field $f$ is re-assigned at statement $s$,
  at the program point before $s$, $f$'s must-call obligations must have been satisfied.
\end{itemize}
\noindent
The first rule ensures that \<close()> is called after the last call
to \<reconnect()>, and the second rule ensures that \<reconnect()>
safely closes \<sock> before re-assigning it. Because the Called Methods Checker
treats calls to an \<@CreatesMustCallFor> method like \<reconnect()> as if the call might
cause arbitrary side-effects, after such a call the only method known to have been
definitely called is the \<@CreatesMustCallFor> method: previous called
methods (including \<close()>) do not appear in the \<@CalledMethods> type qualifier.

% TODO: should this section also include text about unconnected sockets, or is what's here sufficient?


\sectionAndLabel{Ignored exception types}{resource-leak-ignored-exceptions}

The Resource Leak Checker checks that an element's must-call obligations
are fulfilled when that element may go out of scope: at the end of its
lexical scope or when control may be transferred to the end of its lexical
scope, such as via a \<break> or \<continue> statement or via throwing an
exception.  As an example of an exception, consider the following method:

\begin{verbatim}
  void foo() {
    Socket s = ...;
    bar();
    s.close();
  }
\end{verbatim}

If \<bar> is declared to throw an exception, the Resource Leak Checker
warns that a \<Socket> may be leaked.  If \<bar> throws an exception, the
only reference to \<s> is lost, which could lead to a resource leak.

The Resource Leak Checker ignores control flow due to some exceptions.

\begin{itemize}
\item
The Resource Leak Checker ignores run-time errors that can occur
unpredictably at most points in the program. For example, the JVM can throw
an \<OutOfMemoryError> on any allocation.  Similarly,
\<ClassCircularityError>, \<ClassFormatError>, and \<NoClassDefFoundError>
may occur at any reference to a class.  Such exceptions usually terminate
the program.  Accounting for such exceptions would lead to vast numbers of
false positive warnings, so the Resource Leak Checker assumes they are
never thrown.  Strictly speaking, this is an unsoundness.

\item
The Resource Leak Checker also ignores exception types that can be verified
to never occur.  In particular, the Resource Leak Checker ignores \<NullPointerException>s
(use the Nullness Checker, \chapterpageref{nullness-checker}) and
\<ArrayIndexOutOfBoundsException>s and \<NegativeArraySizeException>s (use the Index
Checker, \chapterpageref{index-checker}). Other exception types may be added to this
list in the future.  Please let us know if there is a type that you think should
be ignored by filing an issue listing both the exception type and the
verification tool.
\end{itemize}


\sectionAndLabel{Errors about field initialization}{resource-leak-field-initialization}

% Arguably, this is working around a bug in the
% MustCallConsistencyAnalyzer, which could be improved to avoid issuing
% these false positive warnings.

The Resource Leak Checker warns about re-assignments to owning fields,
because the value that was overwritten might not have had its obligations
satisfied.  Such a warning is not necessary on the first assignment to a
field, since the field had no content before the assignment.  Sometimes,
the Resource Leak Checker is unable to determine that an assignment is the
first one, so it conservatively assumes the assignment is a re-assignment
and issues an error.

One way to prevent this false positive warning is to declare the field as \<final>.

Alternately, to suppress all warnings related to field assignments in the
constructor and in initializer blocks, pass the
\<-ApermitInitializationLeak> command-line argument.  This makes the
checker unsound:  the Resoure Leak Checker will not warn if the constructor
and initializers set a field more than once.  The amount of leakage is
limited to how many times the field is set.


% LocalWords:  de subchecker OutputStream MustCall MustCallUnknown RAII Un
% LocalWords:  PolyMustCall InheritableMustCall MultiRandSelector callsite
% LocalWords:  Partitioner CalledMethods AnoLightweightOwnership NotOwning
% LocalWords:  AnoResourceAliasing MustCallAlias AnoCreatesMustCallFor
% LocalWords:  EnsuresCalledMethods CreatesMustCallFor expr Verification''
% LocalWords:  closeSocket destructor'' destructor BufferedOutputStream
% LocalWords:  AnoResourceAliases createsmustcallfor SocketContainer
% LocalWords:  CalledMethodsBottom ArrayIndexOutOfBoundsException
% LocalWords:  NegativeArraySizeException
